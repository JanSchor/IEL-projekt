
% https://tinyurl.com/2bycrad2

\section{Příklad 5}

\patyZadani{F}

Hodnoty v obvodu dokážeme popsat třemi rovnicemi:
\nopagebreak
\[
\begin{array}{l}
i = \frac{U_R}{R} \implies u_R = R \cdot i\\
U = u_R + u_L \\
i' = \frac{u_L}{L}; \quad i(0) = i_{LP}
\end{array}
\]

Dosadíme vyjádřené \( u_R \) z první rovnice do druhé.
\nopagebreak
\[
U = R \cdot i + u_L \implies u_L = U - R \cdot i
\]

Hodnotu \( u_L \) následně můžeme dosadt do 3. rovnice:
\nopagebreak
\[
\begin{aligned}
i' &= \frac{U - i \cdot R}{L} \\
L \cdot i' &= U - i \cdot R \\
U &= L \cdot i' + R \cdot i; \quad i(0) = i_{LP}
\end{aligned}
\]

Nyní máme vytvořenou diferenciální rovnici.
Sestavíme si charakteristickou rovnici.
\nopagebreak
\[
\begin{array}{l}
i' = \lambda; \quad
i = 1
\end{array}
\]

Dosadíme zaměněné hodnoty pro \( i' \) a i do diferenciální rovnice a vyjádříme si \( \lambda \).
\nopagebreak
\[
L \cdot \lambda + R = 0 \implies \lambda = -\frac{R}{L}
\]

Vypíšeme si očekávané řešení a dosadíme za \( \lambda \) hodnoty, které jsme si vyjádřili.
\nopagebreak
\[
\begin{array}{l}
i(t) = k(t) \cdot e^{\lambda \cdot t} \\
i(t) = k(t) \cdot e^{-\frac{R}{L} \cdot t}
\end{array}
\]
\newpage
Nyní si rovnici zderivujeme a dosadíme ji do diferenciální rovnice. Takhle vytvořenou rovnici potom integrujeme.
\nopagebreak
\[
\renewcommand{\arraystretch}{1.5}
\begin{array}{l}
i'(t) = k'(t) \cdot e^{\lambda \cdot t} \\
k'(t) = \frac{U}{L} \cdot e^{\frac{R}{L} \cdot t} \\
k(t) = \frac{\frac{U}{L}}{\frac{R}{L}} \cdot e^{\frac{R}{L} \cdot t} + k \\
k(t) = \frac{U}{R} \cdot e^{\frac{R}{L} \cdot t} + k
\end{array}
\]

Dosadíme \( k(t) \) do očekávaného řešení.
\nopagebreak
\[
\renewcommand{\arraystretch}{1.5}
\begin{array}{l}
i_L = (\frac{U}{R} \cdot e^{\frac{R}{L} \cdot t} + k) \cdot e^{-\frac{R}{L} \cdot t} \\
i_L = \frac{U}{R} \cdot e^{\frac{R}{L} \cdot t} \cdot e^{-\frac{R}{L} \cdot t} + k \cdot e^{-\frac{R}{L} \cdot t} \\
i_L = \frac{U}{R} + k \cdot e^{-\frac{R}{L} \cdot t}
\end{array}
\]
Dosadíme hodnoty pro t = 0 a dopočítáme derivační konstantu:
\nopagebreak
\[
\begin{array}{l}
8 = \frac{25}{50} + k \cdot e^{-\frac{50}{10} \cdot 0} \\
k = 7.5
\end{array}
\]
Jakmile známe derivační konstantu, jsme schopni vytvořit funkci pro výpočet \( i_L \) v závislosti na čase t.
Dosadíme tedy zadané hodnoty.
\nopagebreak
\[
i_L = \frac{1}{2} + 7.5 \cdot e^{-5t}
\]

Díky Ohmově zákonu víme, že by se měl proud v obvodu přibližovat k 500 mA.
Pojďme si tedy dosadit hodnoty pro t = 0; t = 0.2; t = 0.5; t = 1 a t = 100:
\nopagebreak
\[
\renewcommand{\arraystretch}{1.5}
\begin{array}{l}
i_L(0) = \frac{1}{2} + 7.5 \cdot e^{-5 \cdot 0} = 8 \, \mathrm{A} \\
i_L(0.2) = \frac{1}{2} + 7.5 \cdot e^{-5 \cdot 0.2} = 3.2591 \, \mathrm{A} \\
i_L(0.5) = \frac{1}{2} + 7.5 \cdot e^{-5 \cdot 0.5} = 1.1156 \, \mathrm{A} \\
i_L(1) = \frac{1}{2} + 7.5 \cdot e^{-5 \cdot 1} = 0.5505 \, \mathrm{A} \\
i_L(100) = \frac{1}{2} + 7.5 \cdot e^{-5 \cdot 100} = 0.5 \, \mathrm{A}
\end{array}
\]

Tímto jsme si ověřili, že vzoreček pro funkci funguje.