
\section{Příklad 2}

\druhyZadani{F}

Zjednodušíme rezistory \( R_1 \) a \( R_2 \) do jednoho rezistory \( R_{12} \).
\nopagebreak
\[
R_{12} = R_1 + R_2
\]
\nopagebreak

To stejné uděláme pro rezistory \( R_3 \) a \( R_4 \).
\nopagebreak
\[
R_{34} = \frac{R_3 \cdot R_4}{R_3 + R_4}
\]

Zjednodušený obvod potom bude vypadat takhle:
\par
\nopagebreak
\begin{circuitikz} \draw
(0, 0)
to[V, v<=$U_1$] (0, 2)
to[generic, l=$R_{12}$] (3, 2)
to[generic, l=$R_5$] (6, 2)
to[V, v>=$U_2$] (6, 0)
-- (0, 0)
(3, 2)
to[short, *-, i_=$I_{R34}$] (3, 1.5)
to[generic, -*, l=$R_{34}$, v>=$U_{R34}$] (3, 0)
;
\end{circuitikz}
\par
\vspace{0.3cm}
Na zjednodušeném obvodu již není samotný rezistor \( R_4 \), u kterého potřebujeme zjistit napětí a proud.
Místo toho zjistíme tyto hodnoty na rezistoru \( R_{34} \) a z nich poté odvodíme požadované výsledky pro \( R_4 \).
\par
Jako další krok odpojíme zátěž a zkratujeme zdroje napětí, abychom mohli vypočítat vnitřní odpor skutečného zdroje napětí.
Určíme si takto 2 body A a B, mezi kterými byl rezistor \( R_{34} \).
\par
\nopagebreak
\begin{circuitikz}
\draw
(0, 0)
-- (0, 2)
to[generic, l=$R_{12}$] (3, 2)
to[generic, l=$R_5$] (6, 2)
-- (6, 0)
-- (0, 0)
(3, 2) to[short, *-o] (3, 1.5) node[anchor=west] {A}
(3, 0) to[short, *-o] (3, 0.5) node[anchor=west] {B}
;

\draw
(9, 1) to[short, *-o] (8, 1) node[anchor=east] {A}
(9, 0) -- (9, 2)
(9, 2) to[generic, l=$R_{12}$] (12, 2)
(9, 0) to[generic, l=$R_5$] (12, 0)
(12, 0) -- (12, 2)
(12, 1) to[short, *-o] (13, 1) node[anchor=west] {B}
;
\end{circuitikz}
\par
\vspace{0.3cm}
Vnitřní odpor \( R_i \) je tedy roven celkovému odporu rezistorů \( R_{12} \) a \( R_5 \).
Použijeme tedy vzoreček pro paralelní zapojení rezistorů.
\nopagebreak
\[
R_i = \frac{R_{12} \cdot R_5}{R_{12} + R_5}
\]

\samepage{
Nyní si vytvořím obvod bez rezistoru \( R_{34} \) a pomocí II. K. Z. vypočítám celkový proud protékající tímto obvodem.
Pojmenuji si ho jako proud \( I_x \)
}
\par
\nopagebreak
\begin{circuitikz} \draw
(0, 0)
to[V, v<=$U_1$] (0, 2)
to[generic, l=$R_{12}$] (3, 2)
to[generic, l=$R_5$, i_=$I_x$] (6, 2)
to[V, v>=$U_2$] (6, 0)
-- (0, 0)
(3, 2) to[short, *-o] (3, 1.5) node[anchor=west] {A}
(3, 0) to[short, *-o] (3, 0.5) node[anchor=west] {B}
;
\end{circuitikz}
\par
\vspace{0.3cm}
\[
\begin{aligned}
0 &= I_x \cdot R_{12} + I_x \cdot R_5 + U_2 - U_1 \\
0 &= I_x(R_5 + R_{12}) + U_2 - U_1 \\
I_x &= \frac{U_1 - U_2}{R_5 + R_{12}}
\end{aligned}
\]
\par
Nyní si podle II. K. Z. vypočítáme napětí ve smičce a získáme tak napětí \( U_i \) mezi body A a B.
\par
\nopagebreak
\begin{circuitikz} \draw
(0, 0)
to[V, v<=$U_1$] (0, 2)
to[generic, l=$R_{12}$] (3, 2)
to[generic, l=$R_5$, i_=$I_x$] (6, 2)
to[V, v>=$U_2$] (6, 0)
-- (0, 0)
(3, 2) to[short, *-o] (3, 1.5) node[anchor=west] {A}
(3, 0) to[short, *-o] (3, 0.5) node[anchor=west] {B}
;
\draw[->] (2.8, 1.5) to[short, l_=$U_i$] (2.8, 0.5);
\draw[->] (1.2, 1.35) arc(110:-200:5mm);
\end{circuitikz}
\[
\begin{aligned}
0 &= U_i - U_1 - I_x \cdot R_{12} \\
U_i &= U_1 + R_{12} \cdot I_x
\end{aligned}
\]
\par
Obvod si nyní dokážeme překreslit na jeho variantu se skutečným zdrojem a zátěží.
\par
\nopagebreak
\begin{circuitikz} \draw
(0, 0)
to[V, v<=$U_i$] (0, 2)
to[generic, l=$R_i$, -o, i_=$I_{R34}$] (3, 2) node[anchor=west] {A}
to[generic, l=$R_{34}$, o-o] (3, 0) node[anchor=west] {B}
to[short, o-] (0, 0)
;
\end{circuitikz}
\par
\vspace{0.3cm}
Z tohoto obvodu nyní dokážeme dopočítat proud a napětí na rezistoru \( R_{34} \).
\nopagebreak
\[
\begin{array}{l}
I_{R34} = \frac{U_i}{R_i + R_{34}} \\
U_{R34} = R_{34} \cdot I_{R34}
\end{array}
\]
\par
\nopagebreak
Nyní se vrátíme k obvodu před zjednodušením.
Jelikož jsou rezistory \( R_3 \) a \( R_4 \) zapojeny paralelně, napětí na nich bude stejné jako na rezistoru \( R_{34} \).
Jediné, co je tedy třeba dopočítat, je proud na \( R_4 \), který vypočítáme pomocí vzorečku.
\[
\begin{array}{l}
U_{R4} = U_{R34} \\
I_{R4} = \frac{U_{R34}}{R_4}
\end{array}
\]
Po dosazení hodnot vypočítáme konkrétní hodnoty \( U_{R4} \) \( I_{R4} \).
\nopagebreak
\[
\begin{array}{l}
U_{R4} = 118.0477 \, \mathrm{V} \\
I_{R4} = 181.6119 \, \mathrm{mA}
\end{array}
\]