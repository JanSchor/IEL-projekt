\section{Příklad 4}

\ctvrtyZadani{F}
\nopagebreak
Vytvořím si proudové smyčky a vytvořím si soustavu tří lineárních rovnic.

\begin{align*}
i_A: & \quad
Z_{C1} \cdot I_A + R_1 \cdot I_A + U_1 + Z_{L2} \cdot (I_A - I_C) - U_2 = 0 \\
i_B: & \quad
Z_{L1} \cdot I_B + Z_{C2} \cdot (I_B - I_C) - U_1 = 0 \\
i_C: & \quad
R_2 \cdot I_C + Z_{L2} \cdot (I_C - I_A) + Z_{C2} \cdot (I_C - I_B) = 0
\end{align*}

\begin{align*}
i_A: & \quad
Z_{C1} \cdot I_A + R_1 \cdot I_A + Z_{L2} \cdot I_A + 0 \cdot I_B - Z_{L2} \cdot I_C = U_2 - U_1 \\
i_B: & \quad
0 \cdot I_A + Z_{L1} \cdot I_B + Z_{C2} \cdot I_B - Z_{C2} \cdot I_C = U_1 \\
i_C: & \quad
-Z_{L2} \cdot I_A - Z_{C2} \cdot I_B + R_2 \cdot I_C + Z_{L2} \cdot I_C + Z_{C2} \cdot I_C = 0
\end{align*}

\begin{align*}
i_A: & \quad
(Z_{C1} + R_1 + Z_{L2})I_A + 0I_B - Z_{L2} \cdot I_C = U_2 - U_1 \\
i_B: & \quad
0I_A + (Z_{L1} + Z_{C2})I_B - Z_{C2} \cdot I_C = U_1 \\
i_C: & \quad
-Z_{L2} \cdot I_A - Z_{C2} \cdot I_B + (R_2 + Z_{L2} + Z_{C2}) \cdot I_C = 0
\end{align*}
\par
Nyní převedeme rovnice do matice a získáme tak hodnoty \( I_A \), \( I_B \) a \( I_C \):
\nopagebreak
\begin{align*}
\begin{bmatrix}
Z_{C1} + R_1 + Z_{L2} & 0 & -Z_{L2} \\
0 & Z_{L1} + Z_{C2} & -Z_{C2} \\
-Z_{L2} & -Z_{C2} & R_2 + Z_{L2} + Z_{C2}
\end{bmatrix}
\begin{bmatrix}
I_A \\
I_B \\
I_C
\end{bmatrix}
=
\begin{bmatrix}
U_2 - U_1 \\
U_1 \\
0
\end{bmatrix}
\end{align*}
\newpage

Proud \( I_B \) je roven proudu na cívce \( L_1 \).
Vypočítáme si tedy velikost tohoto proudu a z něj dopočítáme napětí.
\nopagebreak

\[
\begin{array}{l}
I_{L1} = I_B \\
U_{L1} = I_{L1} \cdot Z_{L1} \\
|U_{L1}| = 1.3508 \, \mathrm{V}
\end{array}
\]
\par

Dopočítáme fázový posuv pro \( \varphi_{L1} \).
Za A a B si dosadíme reálnou a imaginární část proudu protékajícího \( L_1 \).
\[
\begin{array}{l}
\varphi_{L1} = arctan(\frac{A}{B}) \\
\varphi_{L1} = -1.4730 rad
\end{array}
\]