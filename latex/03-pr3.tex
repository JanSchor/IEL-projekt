\section{Příklad 3}

\tretiZadani{A}

Tvorba rovnice pro každý uzel (A, B, C) podle I. K.Z.:
\nopagebreak
\begin{align*}
A: & \quad
\frac{U_1 - U_A}{R_1} - \frac{U_A + U_2 - U_B}{R_3} + \frac{U_B - U_A}{R_4} - \frac{U_A}{R_2} = 0 \\
B: & \quad \frac{U_A + U_2 - U_B}{R_3} + I - \frac{U_B - U_C}{R_6} - \frac{U_B - U_A}{R_4} = 0 \\
C: & \quad \frac{U_B - U_C}{R_6} - I - \frac{U_C}{R_5} = 0
\end{align*}

Úprava rovnic pro dosazení do matice:
\nopagebreak
\begin{align*}
A: & \quad
- (\frac{1}{R_1} + \frac{1}{R_3} + \frac{1}{R_4} + \frac{1}{R_2})U_A
+ (\frac{1}{R_3} + \frac{1}{R_4})U_B
+ 0 U_C
= \frac{U_2}{R_3} - \frac{U_1}{R_1} \\
B: & \quad
(\frac{1}{R_3} + \frac{1}{R_4})U_A
- (\frac{1}{R_3} + \frac{1}{R_4} + \frac{1}{R_6})U_B
+ \frac{1}{R_6} U_C
= -I-\frac{U_2}{R_3} \\
C: & \quad
0 U_A
+ \frac{1}{R_6}U_B
- (\frac{1}{R_5} + \frac{1}{R_6})U_C
= I \\
\end{align*}

Maticový tvar rovnic:
\nopagebreak
\begin{align*}
\begin{bmatrix}
-\left(\frac{1}{R_1} + \frac{1}{R_3} + \frac{1}{R_4} + \frac{1}{R_2}\right) & \frac{1}{R_3} + \frac{1}{R_4} & 0 \\
\frac{1}{R_3} + \frac{1}{R_4} & -\left(\frac{1}{R_3} + \frac{1}{R_4} + \frac{1}{R_6}\right) & \frac{1}{R_6} \\
0 & \frac{1}{R_6} & -\left(\frac{1}{R_5} + \frac{1}{R_6}\right)
\end{bmatrix}
\begin{bmatrix}
U_A \\ U_B \\ U_C
\end{bmatrix}
=
\begin{bmatrix}
\frac{U_2}{R_3} - \frac{U_1}{R_1} \\
-I - \frac{U_2}{R_3} \\
I
\end{bmatrix}
\end{align*}

Po dosazení hodnot získáme výsledek \( U_A \), \( U_B \), \( U_C \):
\nopagebreak
\[
\begin{array}{l}
U_A = 49.2546 \, \mathrm{V} \\
U_B = 59.9700 \, \mathrm{V} \\
U_C = 10.5480 \, \mathrm{V}
\end{array}
\]

Napětí mezi uzlem \( U_A \) a stanoveným referenčním uzlem je rovno napětí na rezistoru \( R_2 \).

Pomocí Ohmova zákona dokážeme vypočítat proud \( I_{R2} \):

\nopagebreak
\[
\begin{array}{l}
I_{R2} = \frac{U_A}{R_2} \\
I_{R2} = 1.0052 \, \mathrm{A} \\
U_{R2} = U_A = 49.2546 \, \mathrm{V}
\end{array}
\]