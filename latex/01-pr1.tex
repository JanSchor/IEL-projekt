\section{Příklad 1}

\prvniZadani{F}
Rezistory \( R_2 \), \( R_3 \) a \( R_4 \) tvoří trojúhelník.
Použijeme převod a nahradíme je rezistory \( R_A \), \( R_B \) a \( R_C \).
\par
\nopagebreak
\vspace{0.3cm}
\begin{circuitikz}
\draw
(0, 1.5) node[anchor=east] {A}
to[generic, o-o, l=$R_2$] (3, 3) node[anchor=west] {B}
to[generic, o-o, l=$R_4$] (3, 0) node[anchor=west] {C}
to[generic, o-o, l_=$R_3$] (0, 1.5)
;
\draw[-Latex] (5, 1.5) -- (7, 1.5);

\draw
(9, 1.5) node[anchor=east] {A}
to[generic, o-*, l=$R_A$] (11, 1.5)
to[generic, *-o, l=$R_B$] (12.5, 3) node[anchor=west] {B}
(11, 1.5) to[generic, *-o, l=$R_C$] (12.5, 0) node[anchor=west] {C}
;
\end{circuitikz}
\par
\vspace{0.3cm}
Nahradíme tedy rezistory v obvodu za \( R_A \), \( R_B \) a \( R_C \).
Při této příležitosti také zjednodušíme některé části obvodu.
Zdroje \( U_1 \) a \( U_2 \) nahradíme jedním zdrojem \( U_{12} \).
Rezistory \( R_5 \) a \( R_6 \) nahradíme jedním rezistorem \( R_{56} \).
\par
\nopagebreak
\begin{circuitikz}
\draw
(0, 0)
to[V, v<=$U_{12}$] (0, 2)
to[generic, l=$R_1$] (1.3, 2)
to[generic, -*, l=$R_A$] (3, 2)
(3, 1.3) -- (3, 2.8)
to[generic, l=$R_B$] (4.5, 2.8)
to[generic, l=$R_{56}$] (6, 2.8)
(3, 1.3)
to[generic, l=$R_C$] (4.5, 1.3)
to[generic, -*, l=$R_7$] (6, 1.3)
(6, 2.8) -- (6, 0)
to[generic, l=$R_8$] (0, 0)
;
\end{circuitikz}
\nopagebreak
\[
\begin{array}{l}
U_{12} = U_1 + U_2 \\
R_{56} = \frac{R_5 \cdot R_6}{R_5 + R_6}
\end{array}
\]
\[
\begin{array}{l}
R_a = \frac{R_2 R_3}{R_2 + R_3 + R_4} \quad
R_b = \frac{R_2 R_4}{R_2 + R_3 + R_4} \quad
R_c = \frac{R_3 R_4}{R_2 + R_3 + R_4} \\
\end{array}
\]
\par
\nopagebreak
Opět zjednodušíme obvod a překreslíme všechny sériově zapojené rezistory do jednoho.
\par
\nopagebreak
\begin{circuitikz}
\draw
(0, 0)
to[V, v<=$U_{12}$] (0, 2)
to[generic, -*, l=$R_{1A}$] (3, 2)
(3, 1.3) -- (3, 2.8)
to[generic, l=$R_{B56}$] (6, 2.8)
(3, 1.3)
to[generic, -*, l=$R_{C7}$] (6, 1.3)
(6, 2.8) -- (6, 0)
to[generic, l=$R_8$] (0, 0)
;
\end{circuitikz}
\nopagebreak
\[
\begin{array}{l}
R_{1A} = R_1 + R_A \\
R_{B56} = R_B + R_{56} \\
R_{C7} = R_C + R_7
\end{array}
\]
\par
\nopagebreak
Zjednodušíme paralelně zapojené rezistory do jednoho.
\par
\nopagebreak
\begin{circuitikz}
\draw
(0, 0)
to[V, v<=$U_{12}$] (0, 2)
to[generic, l=$R_{1A}$] (3, 2)
to[generic, l=$R_{B56C7}$] (3, 0)
to[generic, l=$R_8$] (0, 0)
;
\end{circuitikz}
\nopagebreak
\[
R_{B56C7} = \frac{R_{B56} \cdot R_{C7}}{R_{B56} + R_{C7}}
\]
\par
\nopagebreak
Nyní jsme schopni vypočítat celkový odpor \( R_{ekv} \).
Následně vypočítáme celkový proud, který protéká obvodem.
\[
R_{ekv} = R_{1A} + R_{B56C7} + R_7 \\
\]
\[
I = \frac{U_{12}}{R_{ekv}}
\]
Spočítáme napětí na rezistoru \( R_{B56C7} \) a pomocí toho získáme proud protékající rezistorem \( R_{C7} \).
\[
\begin{array}{l}
U_{RB56C7} = R_{B56C7} \cdot I \\
I_{RC7} = \frac{U_{RB56C7}}{R_{C7}}
\end{array}
\]
\par
\nopagebreak
Cílem úlohy je získat hodnotu na rezistoru \( R_3 \).
Pro její výpočet využijeme II. K. Z.
Potřebujeme tedy zjistit napětí na všech rezistorech ve smyčce obsahující rezistor \( R_3 \).
\[
\begin{array}{l}
U_{R1} = I \cdot R_1 \\
U_{R7} = I_{RC7} \cdot R_7 \\
U_{R8} = I \cdot R_8
\end{array}
\]
\par
\nopagebreak
Známe vše potřebné pro dopočítání napětí na \( R_3 \).
Po vypočítání napětí dokážeme spočítat i proud protékající tímto rezistorem.
\[
U_{R3} = U_{12} - U_{R1} - U_{R7} - U_{R8} \\
\]
\[
I_{R3} = \frac{U_{R3}}{R_3}
\]
\par
\nopagebreak
Po dosazení hodnot získáme výsledky:
\[
\begin{array}{l}
U_{R3} = 42.1208 \, \mathrm{V} \\
I_{R3} = 76.5833 \, \mathrm{mA}
\end{array}
\]