%%%%% Projekt z předmětu IEL %%%% 
%% Autor: Jan Schoř, xschorj00

% pdflatex 00-projekt.tex xschorj00.pdf

\documentclass[]{fitiel} % dokumenty v češtině, implicitní

% implicitní cesta k obrázkům
\graphicspath{{fig/}}

% hlavička 
% příkaz logo - vkládá logo fakulty dle zvoleného jazyka
\title{\logo\\IEL -- protokol k projektu}
\author{Jan Schoř \\ xschorj00}
\date{\today} % today - dnešní datum

\begin{document}
	\maketitle
	
	\tableofcontents
	
	\newpage

	%% vložení jednotlivých příkladů
	\section{Příklad 1}


\prvniZadani{F} \newpage
	\section{Příklad 2}

\druhyZadani{F} \newpage
	\section{Příklad 3}

\tretiZadani{A}

Tvorba rovnice pro každý uzel (A, B, C) podle I. K.Z.:
\nopagebreak
\begin{align*}
A: & \quad
\frac{U_1 - U_A}{R_1} - \frac{U_A + U_2 - U_B}{R_3} + \frac{U_B - U_A}{R_4} - \frac{U_A}{R_2} = 0 \\
B: & \quad \frac{U_A + U_2 - U_B}{R_3} + I - \frac{U_B - U_C}{R_6} - \frac{U_B - U_A}{R_4} = 0 \\
C: & \quad \frac{U_B - U_C}{R_6} - I - \frac{U_C}{R_5} = 0
\end{align*}

Úprava rovnic pro dosazení do matice:
\nopagebreak
\begin{align*}
A: & \quad
- (\frac{1}{R_1} + \frac{1}{R_3} + \frac{1}{R_4} + \frac{1}{R_2})U_A
+ (\frac{1}{R_3} + \frac{1}{R_4})U_B
+ 0 U_C
= \frac{U_2}{R_3} - \frac{U_1}{R_1} \\
B: & \quad
(\frac{1}{R_3} + \frac{1}{R_4})U_A
- (\frac{1}{R_3} + \frac{1}{R_4} + \frac{1}{R_6})U_B
+ \frac{1}{R_6} U_C
= -I-\frac{U_2}{R_3} \\
C: & \quad
0 U_A
+ \frac{1}{R_6}U_B
- (\frac{1}{R_5} + \frac{1}{R_6})U_C
= I \\
\end{align*}

Maticový tvar rovnic:
\nopagebreak
\begin{align*}
\begin{bmatrix}
-\left(\frac{1}{R_1} + \frac{1}{R_3} + \frac{1}{R_4} + \frac{1}{R_2}\right) & \frac{1}{R_3} + \frac{1}{R_4} & 0 \\
\frac{1}{R_3} + \frac{1}{R_4} & -\left(\frac{1}{R_3} + \frac{1}{R_4} + \frac{1}{R_6}\right) & \frac{1}{R_6} \\
0 & \frac{1}{R_6} & -\left(\frac{1}{R_5} + \frac{1}{R_6}\right)
\end{bmatrix}
\begin{bmatrix}
U_A \\ U_B \\ U_C
\end{bmatrix}
=
\begin{bmatrix}
\frac{U_2}{R_3} - \frac{U_1}{R_1} \\
-I - \frac{U_2}{R_3} \\
I
\end{bmatrix}
\end{align*}

Po dosazení hodnot získáme výsledek \( U_A \), \( U_B \), \( U_C \):
\nopagebreak
\[
\begin{array}{l}
U_A = 49.2546 \, \mathrm{V} \\
U_B = 59.9700 \, \mathrm{V} \\
U_C = 10.5480 \, \mathrm{V}
\end{array}
\]

Napětí mezi uzlem \( U_A \) a stanoveným referenčním uzlem je rovno napětí na rezistoru \( R_2 \).

Pomocí Ohmova zákona dokážeme vypočítat proud \( I_{R2} \):

\nopagebreak
\[
\begin{array}{l}
I_{R2} = \frac{U_A}{R_2} \\
I_{R2} = 1.0052 \, \mathrm{A} \\
U_{R2} = U_A = 49.2546 \, \mathrm{V}
\end{array}
\] \newpage
	\section{Příklad 4}

\ctvrtyZadani{F}
\nopagebreak
Vytvořím si proudové smyčky a vytvořím si soustavu tří lineárních rovnic.

\begin{align*}
i_A: & \quad
Z_{C1} \cdot I_A + R_1 \cdot I_A + U_1 + Z_{L2} \cdot (I_A - I_C) - U_2 = 0 \\
i_B: & \quad
Z_{L1} \cdot I_B + Z_{C2} \cdot (I_B - I_C) - U_1 = 0 \\
i_C: & \quad
R_2 \cdot I_C + Z_{L2} \cdot (I_C - I_A) + Z_{C2} \cdot (I_C - I_B) = 0
\end{align*}

\begin{align*}
i_A: & \quad
Z_{C1} \cdot I_A + R_1 \cdot I_A + Z_{L2} \cdot I_A + 0 \cdot I_B - Z_{L2} \cdot I_C = U_2 - U_1 \\
i_B: & \quad
0 \cdot I_A + Z_{L1} \cdot I_B + Z_{C2} \cdot I_B - Z_{C2} \cdot I_C = U_1 \\
i_C: & \quad
-Z_{L2} \cdot I_A - Z_{C2} \cdot I_B + R_2 \cdot I_C + Z_{L2} \cdot I_C + Z_{C2} \cdot I_C = 0
\end{align*}

\begin{align*}
i_A: & \quad
(Z_{C1} + R_1 + Z_{L2})I_A + 0I_B - Z_{L2} \cdot I_C = U_2 - U_1 \\
i_B: & \quad
0I_A + (Z_{L1} + Z_{C2})I_B - Z_{C2} \cdot I_C = U_1 \\
i_C: & \quad
-Z_{L2} \cdot I_A - Z_{C2} \cdot I_B + (R_2 + Z_{L2} + Z_{C2}) \cdot I_C = 0
\end{align*}
\par
Nyní převedeme rovnice do matice a získáme tak hodnoty \( I_A \), \( I_B \) a \( I_C \):
\nopagebreak
\begin{align*}
\begin{bmatrix}
Z_{C1} + R_1 + Z_{L2} & 0 & -Z_{L2} \\
0 & Z_{L1} + Z_{C2} & -Z_{C2} \\
-Z_{L2} & -Z_{C2} & R_2 + Z_{L2} + Z_{C2}
\end{bmatrix}
\begin{bmatrix}
I_A \\
I_B \\
I_C
\end{bmatrix}
=
\begin{bmatrix}
U_2 - U_1 \\
U_1 \\
0
\end{bmatrix}
\end{align*}
\newpage

Proud \( I_B \) je roven proudu na cívce \( L_1 \).
Vypočítáme si tedy velikost tohoto proudu a z něj dopočítáme napětí.
\nopagebreak

\[
\begin{array}{l}
I_{L1} = I_B \\
U_{L1} = I_{L1} \cdot Z_{L1} \\
|U_{L1}| = 1.3508 \, \mathrm{V}
\end{array}
\]
\par

Dopočítáme fázový posuv pro \( \varphi_{L1} \).
Za A a B si dosadíme reálnou a imaginární část proudu protékajícího \( L_1 \).
\[
\begin{array}{l}
\varphi_{L1} = arctan(\frac{A}{B}) \\
\varphi_{L1} = -1.4730 rad
\end{array}
\] \newpage
	\section{Příklad 5}

\patyZadani{F} \newpage
	
	%% vložení tabulky s přehledem výsledků
	\section{Shrnutí výsledků}
    \begin{tabular}{|c|c|c|} \hline 
        \textbf{Příklad} & \textbf{Skupina} & \textbf{Výsledky} \\ \hline
        1 & \prvniSkupina & $U_{R3} = 42.1208 \, \mathrm{V}$ \qquad \qquad $I_{R3} = 76.5833 \, \mathrm{mA}$ \\ \hline
        2 & \druhySkupina & $U_{R4} = 42.1208 \, \mathrm{V}$ \qquad \qquad $I_{R4} = 118.0477 \, \mathrm{mA}$ \\ \hline
        3 & \tretiSkupina & $U_{R2} = 49.2546\, \mathrm{V}$ \qquad \qquad $I_{R2} = 1.0052 \, \mathrm{A}$\\ \hline
        4 & \ctvrtySkupina & $|U_{L_{1}}| = 1.3508 \, \mathrm{V}$ \qquad \qquad $\varphi_{L_{1}} = -1.4730 rad$ \\ \hline
        5 & \patySkupina & $i_L = \frac{1}{2} + 7.5 \cdot e^{-5t}$ \\ \hline
    \end{tabular}

	
\end{document}
